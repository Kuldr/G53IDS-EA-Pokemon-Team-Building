Brief\documentclass[a4paper]{article}
\usepackage{lipsum}
\usepackage{url}
\usepackage{graphicx}
\usepackage{listings}
\lstset{language=Python}
\usepackage[margin=2.5cm]{geometry}
\graphicspath{ {images/} }
\usepackage{afterpage}

%Custom Commands
\newcommand{\Pokemon}{Pok\'{e}mon}
\newcommand{\Pokeapi}{Pok\'{e}api}

\begin{document}

\pagenumbering{gobble}

\begin{center}
	\includegraphics[height=3cm]{UoNLogo.png}

	\vfill
	\Huge{\textbf{Benjamin Joshua Charlton}}
	\vfill
	\LARGE{	ID Number: 4262648 \\
			Supervisor: Professor Robert John \\
			Module Code: G53IDS
			\vfill
			2018/04
			\vfill}
	\afterpage{\null\newpage}
\end{center}


\begin{titlepage}
	\newcommand{\HRule}{\rule{\linewidth}{0.5mm}}
	\center{}

    %Headings
	\includegraphics[height=3cm]{UoNLogo.png}\\[0.5cm]
	\Large{Computer Science with Artificial Intelligence MSci}\\[0.5cm]
	\large{G53IDS - Individual Dissertation}\\[0.5cm]

    %Title
	\HRule{}\\[0.4cm]
	{\huge\bfseries Applying Evolutionary Algorithms to \Pokemon{} Team Building}\\[0.4cm]
	\HRule{}\\[1.5cm]

	Submitted April 2018, in partial fulfilment of the conditions for the award of the degree Computer Science with Artificial Intelligence MSci
	\vfill

    %Authors
	\begin{minipage}{0.4\textwidth}
		\begin{flushleft}
			\large
			\textit{Author}\\
			Benjamin \textsc{Charlton}\\
            psybc3@nottingham.ac.uk\\
            4262648
		\end{flushleft}
	\end{minipage}
    \begin{minipage}{0.4\textwidth}
		\begin{flushright}
			\large
			\textit{Supervisor}\\
			Prof.\@ Bob \textsc{John}\\
            Robert.John@nottingham.ac.uk
		\end{flushright}
	\end{minipage}


	\vfill
	School of Computer Science \\
	University of Nottingham \\

	\vfill
	I hereby declare that this dissertation is all \\
	my own work, except asa indicated in the text:
	\\ [1cm]
	Signature:\rule{6cm}{0.15mm}

    %Date
	\vfill\vfill\vfill
	{\large24\textsuperscript{th} April 2018} %TODO WHAT DATE
	\vfill

\afterpage{\null\newpage}
\normalsize{}
\end{titlepage}


\section*{Abstract}
\begin{itemize}
    \item Short overview of the entire project
\end{itemize}
\section*{Acknowledgements}
\begin{itemize}
    \item Both of these sections shouldn't be proper sections
\end{itemize}
\pagebreak

%Contents Page
\tableofcontents
\setcounter{page}{1}
\pagenumbering{arabic}
\pagebreak

%TODO WORK OUT ALL OF THE SECTIONS
\section{Introduction}
\begin{itemize}
	\item What can be done to improve this section
\end{itemize}
\subsection{Introduction}
\par
Video games are an ever growing field of interest for many people.
Like many games people play competitively against each other and in some cases their are tournaments on an international scale with whole teams set up to win the large prize pools\cite{eSportsPrize}\cite{teamEarnings}.
This field has become to be known as eSports.
\par
Game playing is an obvious application of AI techniques as you can objectively score wins and losses.
Initially this has been seen with traditional board games like Chess\cite{deepBlue} and Go\cite{alphaGo}.
During The International 2017 for DotA 2 this all changed as the worlds of AI and eSports came together in a 1v1 show match\cite{openAI}.
Elon Musk, backer of this initiative, said that this is `Vastly more complex than traditional board games like chess \& Go'\cite{openAI}.
Typically in these set ups their is no decisions to be made prior to the game itself, in fact to limit the DotA 2 AI they limited it to a preselected character to play with.
\par
Many strategy games have a decision making element before the game is played.
Collectable Card Games (CCGs) have the choice of which cards to put in your limited deck or in \Pokemon{} you have to choose which members of your team to use and how you have trained them.
This element of the games is refereed to as deck/team building as the player has to build up what they bring to the game from an empty deck or team.
Some players are very good at playing the game but not very good at choosing which deck/team to bring resulting in a practice called `netdecking', being called so as the player will take someone else's deck/team from the internet instead of coming up with their own.
\par
\Pokemon{} has a tricky team building process that is rather hard for new players to comprehend.
In the build up to the annual \Pokemon{} World Championships, many players will go through the tedious process of team building to make sure they have the right strategies and counters in place to bring to the matches ahead\cite{worldsOverview}.
This requires expert knowledge such as type match ups and speed tiers along side several rules of thumb.
Once a general strategy is in place the players must perform several calculations to optimise the statistics of their team, optimising offensive stats to faint certain threats or optimising defensive stats to allow the team to survive after certain moves are used against them.
\subsection{Motivation}
\par
The motivation for this project comes partly from the novelty of the idea.
As evolutionary algorithms are designed to simulate survival of the fittest could it work on a representation of a biological population.
In the \Pokemon{} games there is an idea that their is a living ecosystem with predators and prey as well as a system for the \Pokemon{} to fight in combat to decided who is the strongest.
\par
Further motivation comes from the seeing how AI game playing is progressing and trying to create something that will combat other aspects that come into playing video games.
This approach could be useful for players of the game allowing them to find new and winning strategies that might not have previously been known.
It could also help the development team behind the games balance the game despite the ever increasing mass of combinations that they have to consider.
\subsection{Aims and Objectives}
\par
This project aims to implement a evolutionary algorithm that will create a \Pokemon{} team.
The output of such will be a team of \Pokemon{}, including all of the vital statistics and moves; this would be dependant of the format.
These results will be then compared to human designed solutions to conclude if the evolutionary algorithm is comparable to an expert.
\\ \\The objectives of this project are:
\begin{enumerate}
    \item Research evolutionary algorithm methods used to approach similar problems.
    Looking in detail about issues such as representation, evaluation and validity of the chromosomes.
    This information will be used to help direct how best to approach the problem and design the system.
    \item Design an effective and efficient way to model and represent the problem in the evolutionary algorithm while still maintaining relevant data to the problem.
    This will also be used to form the output so will need to be readily be able to translate into a readable format for the user to understand.
    \item Develop a Genetic and Memetic algorithm to tackle the problem from scratch, including conventional and unique methods for the various stages in the algorithms.
    Making sure that all of the elements of the code base are created in a fashion that allows for reusability and adaptation.
    For example both the Genetic and Memetic algorithm could share the methods like objective evaluation and selection.
    This will all be developed from scratch (bar the use of a few relevant APIs).
    \item Compare and analyse the results of each evolutionary algorithm (with a variety of settings) with each other and human designed solutions.
    Solutions for comparison will come from readily available teams from top players and analysis will be taken by evaluating the solutions.
    If solutions are viable enough they can be input into the game and used in some real world settings such as battling with other players, rather than being graded by a score.
\end{enumerate}

\section{Related Work}
\begin{itemize}
   \item Investigate more works to talk about here (gaming and more general)
   \item GAs, MAs and other EAs
\end{itemize}
\subsection{AI and Game Playing}
Many AI approaches to games tackle the aspect of playing the matches and the decision making process to choose the best action.
Lots of research and development has happened in these areas with many effective techniques being discovered.
One key reason the problem of prematch decision making hasn't been tackled is due to the lack of need for it with classical table top games requiring no preparation before the match begins.
\\ \par
Chess was an early and significant example in the history of AI, with the Deep Blue computer from IBM successfully beating the at the time world champion Garry Kasparov\cite{deepBlue}.
This was significant as creating a winning chess AI was seen to be the next big milestone at the time in AI\@.
To achieve this Deep Blue used a combination of techniques with the main underlying AI technique being a search method.
To achieve this at the time custom hardware was created to help speed up the computation with a highly parallel structure that could evaluate nodes on the search tree quickly using hardware implementations.
This meant that it could effectively search deeper than any other AI at the time.
\par
Go is the most recent milestone game to be beaten with AlphaGo claiming its victory against one of the best Go players, Lee Sedol, in March 2016\cite{alphaGo}.
To achieve this the team uses Deep Neural Networks combined with Monte Carlo tree search, with a combination of supervised learning from human games and reinforcement learning via self-play.
\par
Recently AlphaGo has been beaten by a variation of itself AlphaGoZero, named as it had learnt from zero human knowledge\cite{alphaGoZero}.
AlphaGoZero learnt entirely from self-play and achieved super human performance.
This shows that not only can AI techniques successfully solve tasks but it is possible to do so with no expert knowledge.
\\ \par
As the field of AI game playing moves forward into more complex games prematch decisions will need to be considered.
With current methods it would be rather simple to build and train an AI to play turn based strategy games, such as collectable card games or in this case \Pokemon{}, but the deck/team building would require an expert to decide what the AI will be trained to use.
This is often problematic as season rotation could add in new elements to the game or make certain elements no longer useable, or shifts in the metagame will mean that the AI is easily countered.
\par
Team building is a form of optimisation problem as you are trying to bring the optimal team to the match so you have the best chance of winning.
\par
A variety of work has been conducted looking at optimisation via AI techniques, in this review of previous works a focus has been upon techniques that tackled having a large, vast search space and where the correctness of a solution was hard to judge.
Both of these issues are problems that will have to be over come in building this project.
\subsection{Hearthstone Deck Building GA}
\par
Garc{\'\i}a-S{\'a}nchez et al.\ tackled a very similar problem using a genetic algorithm to approach deck building\cite{hearthstoneAI}.
The example they used was a popular collectable card game, Hearthstone, and they tried to create a viable competitive deck through the genetic algorithm.
This is of particular interest as several parts of their study directly relate to what the project is trying to achieve, as well as several short comings that will have to take into account.
\par
An evolutionary algorithm was used as `they commonly produce very effective combinations of elements'\cite{hearthstoneAI}, which a deck can be described as a combination of cards.
This also allowed for competitive decks to be built from scratch with no expert knowledge required.
This was done by encoding the individuals as a vector of cards, in this case the 30 element vector became the deck.
\par
In the deck building process it was possible to have a deck that would violate the rules of the game.
To discourage the EA from creating decks that violate these rules; a correctness metric was taking into account when calculating the fitness of each individual.
If an individual was incorrect it wasn't evaluated further in the fitness calculation process and given the worse possible score\cite{hearthstoneAI}.
\\ \par
A large part of the work talks about the fitness evaluation, as with many of these processes it is hard to quantify how well a individual will do.
Even though human experts will have a strong intrinsic idea to what is better or worse, they will often playtest their ideas for several matches, sometimes ranging into the hundreds, before applying some analysis to develop their idea further.
To give the most realistic simulation a separate AI played each individual against some previously expert designed decks, that have been proven to be successful in the metagame that the AI was evolving in.
It played several matches with each individual against the decks, with 16 matches versus the 8 chosen opponent decks totalling 128 matches per individual\cite{hearthstoneAI}.
\par
One of the key upsides of this was that the GA could evolve to combat common decks that it would face if it was to be played in competitive play.
Knowledge of key opposing threats is something that high level experts take into account, this knowledge was taken from an article evaluating the current meta game written by professional players of the game.
Another upside was from having a high number of matches being played for the evaluation; by playing a large proportion of games they can mitigate any randomness in the matches that aren't down to the deck building, such as the cards in the starting hand.
To further mitigate randomness and to give a more accurate representation of the individuals fitness, statistical analysis was used to calculate the final fitness score.
This was done to also as it is important to find `a deck with a fair chance to win against all decks in the metagame' not one that is strong against a particular opponent and weak to all others.
\par
Through this method of fitness evaluation some flaws and potential shortcomings where brought up.
As the fitness evaluation required a large amount of matches to be played it resulted in `every execution of the algorithm requires several days'\cite{hearthstoneAI}, even when running on a small population size and number of generations (10 and 50 respectively).
Although not a specifically a bad thing a longer run time would discourage the use of such tools, more so in games where the meta game changes more rapidly.
The evaluation also required the expert knowledge of the meta decks to be tested against or even that an established meta is in effect where as many experts can theory craft a rough idea before changes to the meta happen.
The most significant flaw with this style of evaluation is that it relies on the ability of another AI to pilot the deck.
It was suggested that it may overfit `with regards to the AI capabilities and playing style' even going as far as pointing out that it `can use some decks (and some playing styles) better than others'\cite{hearthstoneAI}.
\subsection{Evolutionary Algorithms}
Evolutionary Algorithms have been shown to be a strong way to find near optimal solutions to complex problems. Zitzler and Thiele spoke about their merits in their case study suggesting that "Many real-world problems involve simultaneous optimization"\cite{EACaseStudy}. They pointed out that the often some factors would conflict and the EA were strong at finding a balance between them. It was concluded that "All multiobjective EA's clearly outperformed a pure random search"\cite{EACaseStudy}.§
\par
Maumita Bhattacharya spoke about differing approaches to the problem of computationally expensive fitness evaluation\cite{expensiveOptimisation}.
Several of the techniques described approximated the fitness of each individual while only periodically calculating a more accurate value.
Some of these techniques achieved this my simulating a simpler model while more specific techniques where discussed too.
In a population search method like evolutionary algorithms, a technique called Fitness Inheritance could be used.
As the children individuals will closely resemble the parents you could give a rough approximation to how different they actually are and use that difference to evaluate the children.
This will greatly reduce the number of computationally expensive evaluations you need to do with potentially as few as only scoring the initial population.
However it is suggested that you periodically recalculate using the true fitness evaluation as `Using true fitness evaluation along with approximation is thus extremely important to achieve reliable performance'.
\par
Fitness Inheritance was first proposed by Smith et al.\cite{fitnessInheritance} where they applied the idea to a simple problem OneMax.
They investigated two simple methods to inherit fitness.
Averaged Inheritance simply said `a child is assigned the average fitness of its parents', this is extremely cheap computationally and requires no detailed analysis.
Another method was Proportional Inheritance where `the average is weighted based on the amount of genetic material taken from each parent', while slightly more expensive it takes into account of none symmetric crossover schemes such as one point crossover where the genetic material is not equal from each parent.

\section{Description of the Work}
\begin{itemize}
    \item What the project is meant to achieve
	\item How is it meant to function
	\item Brief list of functional requirements
\end{itemize}

\section{Methodology}
\begin{itemize}
    \item Methodology for evaluating results
	\item Expand current sections
\end{itemize}

\section{Design and Implementation}
\begin{itemize}
   \item Description of the designs
   \item How it addresses the problem
   \item Why it is designed that way
   \item Languages/platforms chosen
   \item Problems encountered
   \item Design changes based upon implementation
\end{itemize}

\section{Evaluation}
\begin{itemize}
	\item How the EA was tested
	\item Statistical evaluation of performance
	\item Evaluation of software performance
\end{itemize}

\section{Summary and Reflections}
\begin{itemize}
	\item Project Management \begin{itemize}
		\item Work Plan Reflection
		\item Time and Resource management
	\end{itemize}
	\item Contributions and Reflections \begin{itemize}
		\item Innovation, Creativity and Novelty
		\item Personal reflection on plan and experience
	\end{itemize}
\end{itemize}

\pagebreak
\section{Appendix}
\subsection{Code} %TODO FORMAT AND WORK OUT WHAT CODE TO ADD
%The code is set in these files to allow for them to stay as the snapshot they were
\begin{itemize}
	\item Add in the Code here with the various subsections
\end{itemize}


\subsection{Meeting Minutes}
\begin{itemize}
	%TODO ADD IN ANY FINAL MEETINGS
	\item Add in any final meetings
\end{itemize}
\subparagraph{6th October}
Overview and recap of the project idea, including background and information on the topic.\\
Discussion of the upcoming deliverables and deadlines.\\
Agreed upon a deadline of the 11/10 for the project proposal and ethics form as well as the next meeting to discuss it on the 18/10.\\
\subparagraph{26th October}
Recap of the project proposal and confirmed it has been submitted.\\
Using python as the language to develop the project in
Overview of the next couple of weeks as per the work plan; revision/research of evolutionary algorithms and installing basic software.\\
Brief discussion on the possibility of writing up any research taking place, I will be looking into how to tie this into the interim report and bring up ideas at the next meeting.\\
Agreed to meet on 2/11 at 2pm.\\
\subparagraph{2nd November}
Spoke about the potential of drafting the interim report over the course of the project, agreed to do this continually through out and send the first iteration before the next meeting with sections and some bullet points.\\
Overview of the next stages in the plan.\\
Suggested having the basic GA outline to be running on a benchmark test to prove its working, this is to be shown at the next meeting.\\
Agreed to meet on 21/11 at 12 noon.\\
\subparagraph{21st November}
Reviewed the outline for the Interim Report, suggesting that its good but could potentially merge a few sections together.\\
Demonstrated the GA structure with on the sum of squares benchmark.\\
Discussed writing sections of the interim report now, particularly the related work section. This will be sent before the next meeting.\\
Agreed to meet on 24/11 at 1:30 to review the completed section of the interim report.\\
\subparagraph{28th November}
Reviewed the current draft of the related work section of the interim report, discussing how it can be improved.\\
Agreed to iterate the interim report over email with a final draft being sent by the morning of Thursday the 7th of December.\\
Agreed to meet on 15/12 at 12:00 to discuss the project progression and next steps.\\
\subparagraph{15th December}
Brief Discussion of the Interim Report.\\
Discussed the work to take place over Christmas/Exam Periods.\\
Agreed to meet during the first week of 2nd term, exact times to be confirmed at a later date.\\
\subparagraph{7th February}
Discussed the mark and feedback of the Interim Report.\\
Discussed progress made over the past weeks.\\
Brought up that the project is running slightly behind but it is believed that it can be caught up soon enough.\\
Agreed to meet on 15/2 at 11:00 with a working demo of the code running, this just requires a basic objective function being written.
\subparagraph{15th February}
Discussed the current progress, unable to show a demonstration of the running code due to a bug in the objective function evaluation.\\
Discussed the long running time of the code and how to overcome this for demonstration purposes, agreed that the demo day would work best as a more formal presentation containing the results and that output could be sent via email to show the project running.\\
Agreed to send the output of a working solution once the bug has been fixed, next meeting to be agreed once that has been completed.
\subparagraph{9th March}
Discussed the next stages of the project, prioritising adaption to a memetic algorithm and more advanced methods.\\
Discussed future meetings involving the writing of the dissertation, emails with feedback for each chapter.\\
Creation of class diagrams for the dissertation to show the overall structure of each class and how they interact.\\
Next meeting to be arranged over email as and when required.
\subparagraph{6th April}
Discussed current position of the whole project including dissertation progress
Discussed the based way to approach the running of the program to get data for evaluation.\\
Discussed adapting the benchmark function to show the MA works
Agreed to send the current state of the dissertation, mainly the related work section.\\
Agreed to send first draft before the morning of the 16th April, with most sections fully completed.\\
Next meeting for 16th April at 4pm.

\pagebreak
\subsection{Work Plan} %TODO DO I WANT/NEED TO UPDATE THE WORK PLAN?
\begin{center}
    \includegraphics[height=24.8cm]{workPlan.png}
\end{center}


%Bibliography
\subsection{References}
\bibliography{Dissertation}
\bibliographystyle{plain}


\end{document}
