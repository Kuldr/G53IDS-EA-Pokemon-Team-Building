\documentclass{article}
\usepackage{lipsum}

%Custom Commands
\newcommand{\Pokemon}{Pok\'{e}mon}


\begin{document}

%Title Information
\title{
    Project Proposal
    \\ \large{G53IDS}
    \\ \large{Project Title: Applying Evolutionary Algorithms to \Pokemon{} Team Building}\vspace{-3ex}}
\author{4262648 Benjamin Charlton (psybc3)}
\date{\vspace{-2ex}11\textsuperscript{th} October 2017}
\maketitle

\section{Background and Motivation}
\lipsum[1]

\section{Aims and Objectives}
\lipsum[1]

\section{Work Plan}
To help outline the project flow and I have created a gantt chart (found in the appendix) and a the following descriptions of each portion.

%Explanations of all of the Points on the work plan
\begin{description}
\item [\large{Documentation}]
\item [Project Proposal] Write the project proposal and ethics form for approval of supervisor, due 13\textsuperscript{th} October.
\item [Revised Project Proposal] Any revisions to the project proposal that need to be made after supervisor feedback, due 23\textsuperscript{rd} October
\item [Interim Report] Write the interim report summarising the work so far, due 8\textsuperscript{th} December.
\item [Dissertation] Write the final dissertation, due 24\textsuperscript{th} April.

\item [\large{Research}]
\item [Research EAs] Research into the various Evolutionary Algorithms, like Genetic Algorithms, Memetic Algorithms \& Multimeme Memetic Algorithms.
\item [Review languages for EAs] Primarily seeing if python has suitable features required to make the development of evolutionary algorithms easier or whether Java would be better suited.
\item [Research Representation] Look into various representation methods that could be applied to the problem.

\item [\large{Development}]
\item [Create GA Structure] Set up the basic main loop and the relevant interfaces.
\item [Representation] Code the structure that the model will be represented by to be used as chromosomes in the Evolutionary Algorithm.
\item [Data import] Build methods to import the data that is needed at various points in the running of the algorithm.
\item [Validation Method] Build a method that will check that a chromosome is valid, for use after other methods so that invalid solutions aren't created.
\item [Basic Methods for GA] Create trivial methods for each stage of the Genetic Algorithm to allow for test runs.

\item [\large{Miscellaneous}]
\item [Install Software] Install and configure the relevant software and libraries that are required for the project.

\item [\large{Other Commitments}]
\item [Welcome Week] First week of the year, time set aside to allow for settling in as well as running various welcome events.
\item [Christmas Holiday] Time off after Autumn term, Although not completely set aside allows for some time to relax.
\item [Autumn Exams] Potentially could have exams for the entire 2 weeks so is set aside to allow for revision and the exams themselves.
\item [Easter Holiday] Time off after Spring term, As no teaching is happening will give more time to concentrate on coursework and dissertation.
\end{description}

%Bibliography
\nocite{bar}
\bibliography{ProjectProposal}
\bibliographystyle{plain}

\end{document}
