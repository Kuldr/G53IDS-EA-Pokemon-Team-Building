\documentclass[a4paper]{article}
\usepackage{lipsum}
\usepackage{url}

%Custom Commands
\newcommand{\Pokemon}{Pok\'{e}mon}


\begin{document}

%Title Information
\title{
    Project Proposal
    \\ \large{G53IDS}
    \\ \large{Project Title: Applying Evolutionary Algorithms to \Pokemon{} Team Building}\vspace{-3ex}}
\author{4262648 Benjamin Charlton (psybc3)}
\date{\vspace{-2ex}11\textsuperscript{th} October 2017}
\maketitle

\section{Background and Motivation}
Video games are an ever growing field of interest for many people. Like many games people play competitively against each other and in some cases their are tournaments on an international scale with whole teams set up to win the large prize pools\cite{eSportsPrize}\cite{teamEarnings}. This field has become to be known as eSports.\\
Game playing is an obvious application of AI techniques as you can objectively score wins and losses. Initially this has been seen with traditional board games like Chess\cite{deepBlue} and Go\cite{alphaGo}. During The International 2017 for DotA 2 this all changed as the worlds of AI and eSports came together in a 1v1 show match\cite{openAI}. Elon Musk, backer of this initiative, said that this is `Vastly more complex than traditional board games like chess \& Go'\cite{openAI}.Typically in these set ups their is no decisions to be made prior to the game itself, in fact to limit the DotA 2 AI they limited it to a preselected character to play with.\\
Many strategy games have decision making before the game is played. Collectable Card Games (CCGs) have the choice of which cards to put in your limited deck or in \Pokemon{} you have to choose which members of your team to use and how you have trained them. This element of the games is refereed to as deck/team building as the player has to build up what they bring to the game from an empty deck or team. Some players are very good at playing the game but not very good at choosing which deck/team to bring resulting in a practice called `netdecking', being called so as the player will take someone else's deck/team from the internet instead of coming up with their own.\\
\Pokemon{} has a tricky team building process that is rather hard for new players to comprehend. In the build up to the annual \Pokemon{} World Championships, many players will go through the tedious process of team building to make sure they have the right strategies and counters in place to bring to the matches ahead\cite{worldsOverview}. This requires expert knowledge such as type match ups and speed tiers along side several rules of thumb. Once a general strategy is in place the players must perform several calculations to optimise the statistics of their team, optimising offensive stats to faint certain threats or optimising defensive stats to allow the team to survive after certain moves are used against them.
An attempt at automating the deck building process via evolutionary algorithms have been made in Hearthstone, a popular CCG\cite{hearthstoneAI}. The results where comparable of a well designed human deck which a talent player could then go and take to win tournaments. Another evolutionary algorithm was created to optimise build orders in StarCraft 2, a real time strategy game. The solution that the AI came up with was quicker than the best known strategy at the time and went on to be used in competitive play\cite{starcraftEA}
\Pokemon{} team building is a similar problem to the other deck building problems stated. These where solved to an respectable standard by evolutionary algorithms and as such it seems worthwhile to try and apply this method to the problem.

\section{Aims and Objectives}
\lipsum[1]

\section{Work Plan}
To help outline the project flow and I have created a gantt chart (found in the appendix) and a the following descriptions of each portion.

%Explanations of all of the Points on the work plan
\begin{description}
\item [\large{Documentation}]
\item [Project Proposal] Write the project proposal and ethics form for approval of supervisor, due 13\textsuperscript{th} October.
\item [Revised Project Proposal] Any revisions to the project proposal that need to be made after supervisor feedback, due 23\textsuperscript{rd} October
\item [Interim Report] Write the interim report summarising the work so far, due 8\textsuperscript{th} December.
\item [Dissertation] Write the final dissertation, due 24\textsuperscript{th} April.

\item [\large{Research}]
\item [Research EAs] Research into the various Evolutionary Algorithms, like Genetic Algorithms, Memetic Algorithms \& Multimeme Memetic Algorithms.
\item [Review languages for EAs] Primarily seeing if python has suitable features required to make the development of evolutionary algorithms easier or whether Java would be better suited.
\item [Research Representation] Look into various representation methods that could be applied to the problem.

\item [\large{Development}]
\item [Create GA Structure] Set up the basic main loop and the relevant interfaces.
\item [Representation] Code the structure that the model will be represented by to be used as chromosomes in the Evolutionary Algorithm.
\item [Data import] Build methods to import the data that is needed at various points in the running of the algorithm.
\item [Validation Method] Build a method that will check that a chromosome is valid, for use after other methods so that invalid solutions aren't created.
\item [Basic Methods for GA] Create trivial methods for each stage of the Genetic Algorithm to allow for test runs.

\item [\large{Miscellaneous}]
\item [Install Software] Install and configure the relevant software and libraries that are required for the project.

\item [\large{Other Commitments}]
\item [Welcome Week] First week of the year, time set aside to allow for settling in as well as running various welcome events.
\item [Christmas Holiday] Time off after Autumn term, Although not completely set aside allows for some time to relax.
\item [Autumn Exams] Potentially could have exams for the entire 2 weeks so is set aside to allow for revision and the exams themselves.
\item [Easter Holiday] Time off after Spring term, As no teaching is happening will give more time to concentrate on coursework and dissertation.
\end{description}

%Bibliography
\bibliography{ProjectProposal}
\bibliographystyle{plain}

\end{document}
